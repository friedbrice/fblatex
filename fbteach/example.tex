\documentclass[10pt, letterpaper]{letter}

\usepackage[utf8]{inputenc}
\usepackage[T1]{fontenc}

\usepackage{fbteach}

\setcourse{MATH 201-03}
\setcoursename{Calculus I}
\setterm{Fall 2015}

\setinstructor{Dr. Daniel Brice}
\setinstructorphone{(661) 654-3959}
\setinstructoremail{daniel.brice@csub.edu}

\setfinalday{Mon, Nov 23}
\setfinaltime{2:00 PM -- 4:30 PM}
\setfinalroom{room TBA}

\setmeetingdays{Mon, Fri}
\setmeetingtime{12:45 PM -- 2:00 PM}
\setmeetingroom{Sci III 102}

\setmeetingtwodays{Tue}
\setmeetingtwotime{12:45 PM -- 2:45 PM}
\setmeetingtworoom{Sci III 101}

\setofficehoursdays{Tue, Wed}
\setofficehourstime{10:00 AM -- 11:15 AM}
\setofficehoursroom{Sci III 202}

\setofficehourstwodays{Wed, Thu}
\setofficehourstwotime{12:45 PM -- 2:00 PM}
\setofficehourstworoom{Sci III 202}

\begin{document}

\makesyllabushead

\section{Course Description}

Introduction to the differential calculus of elementary
(including logarithmic, exponential, and trigonometric)
functions. Emphasis on limits, continuity, and differentiation.
Applications of differentiation (including curve sketching,
optimization, and related rates); antiderivatives.

5 Quarter Units.

\section{Course Objectives and Learning Outcomes}

After completing this course, students will:
\begin{itemize}
    \item Understand what a mathematical function is and how
        mathematical functions can be used to model real-world problems.
    \item Represent mathematical functions algebraically, numerically,
        and graphically, and fluidly move between these representations.
    \item Understand what a limit is, know the algebraic and geometric
        properties of limits, an how limits can be used to study
        functions.
    \item Understand what a derivative is, know the algebraic and
        metric properties of derivatives, and how derivatives can be
        used to study functions and solve real-world problems.
    \item Evaluate limits and derivatives and use the results to study
        functions.
\end{itemize}

\section{Required Texts and Materials}

The textbook you need depends on whether you will be going on to
Calculus III. Students planning on taking Calculus III need
\emph{Calculus, Concepts and Contexts} fourth edition by J. Stewart,
ISBN 0-495-55742-0 (bound) or 1-111-02944-X (loose-leaf). Students not
planning on taking Calculus III need \emph{Single Variable Calculus,
Concepts and Contexts} fourth edition by J. Stewart,
ISBN 0-495-55972-5.

Devices with symbolic CAS capabilities, communication capabilities,
touchscreens, or QWERTY keyboards are prohibited. Students should have a
scientific or graphing calculator that does not violate the above
policy, for example, a TI-30 series scientific calculator or a TI-86
graphing calculator.

\section{Attendance}

Attendance is required and contributes to your grade. If you miss an
exam due to medical reasons or a personal emergency, please contact me
as soon as possible after the missed exam to schedule a makeup. Exams
cannot be made up for reasons of convenience or preference or family
vacation. Quizzes and worksheets cannot be made up for any reason.

\section{Reading and Homework}

You are expected to read textbook sections before we meet in class to
discuss those sections. This will make class time more valuable and
help keep you from getting confused.

Homework is assigned for each textbook section. Homework is required,
though it is not collected. Quizzes and exams are based on the homework,
and so completing the homework is the best way to study. Homework must
be completed by the day of the relevant quiz or exam.

Reading and homework assignments for each textbook section, and the
quizzes and exams that will be based on them, are listed on a handout
titled ``Reading and Homework Assignments.''

\newpage

\section{Assessment}

During the course, you will accumulate points by completing various
assessments. The assessments and their point values are listed below,
for a total of 400 points for the course.

\begin{tabular}{l|l|l}
    \textbf{Assessment} & \textbf{Date} & \textbf{Point Value} \\
    \hline
    Attendance and Participation
        & every day & 50 points (maximum) \\
    Quizzes and Worksheets
        & most Tuesdays and as assigned & 100 points (maximum) \\
    Chapter 2 Exam & October 5   &  50 points \\
    Chapter 3 Exam & October 30  &  50 points \\
    Chapter 4 Exam & November 20 &  50 points \\
    Final Exam     & November 23 & 100 points \\
\end{tabular}

During the course, it may be possible to earn more than 50 points from
attendance and participation or more than 100 points from quizzes and
worksheets. Points above the maximum will not contribute to your grade.

\section{Grade Assignment}

Letter grades will be assigned according to the following point cutoffs.

\begin{tabular}{l|c|c|c|c|c|c|c}
    \textbf{Grade}        & A   & A-  & B+  & B   & B-  & C+  & C   \\
    \hline
    \textbf{Point cutoff} & 360 & 342 & 318 & 300 & 282 & 258 & 240 \\
\end{tabular}

Your assigned grade is not negotiable. A grade of D will not be
assigned. A point total below 240 is an F.

\section{Tutoring and Office Hours}

Extra help is available to all students free of charge. The Mathematics
Tutoring Center is located in Science III room 208 offers drop-in and
by-appointment tutoring five days a week. Your instructor's office hours
are available for additional help beyond that. If you cannot make it to
office hours, please email your instructor to make an appointment.

\section{Standard for Academic Integrity}

Students are expected to do all the work assigned to them without
unauthorized assistance and without giving unauthorized assistance. For
the complete policy, see the university catalog. Misrepresenting
someone else's work as your own is dishonest and wrong, and if found
will result in suspension and expulsion.

Quizzes and Exams are to be completed individually. If you suspect
someone is copying your work, please raise your hand and ask to move to
another desk---I do not require you to report the suspected copier.

Worksheets and homework may be completed collaboratively, though each
student must submit their own paper copy.

\section{Accommodations for Students with Disabilities}

To request academic accommodations due to a disability, please contact
the Office of Services for Students with Disabilities (SSD) as soon as
possible. You must have an accommodations letter from the SSD office
documenting that you have a disability; present the letter to me during
my office hours as soon as possible or in the first class period.

\end{document}
